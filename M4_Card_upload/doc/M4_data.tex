\documentclass[11pt,a4paper]{article}

\usepackage{geometry}
\usepackage{times,amsmath,amssymb}
\usepackage{chicago}
\usepackage{graphicx}
\usepackage{xcolor}
\usepackage{booktabs}

\setlength{\textwidth}{17cm}
\setlength{\textheight}{27cm}
\setlength{\oddsidemargin}{0cm}
\setlength{\topmargin}{-2.8cm}

\def\cite{\citeN}%default to author (year)
\def\citecomma#1{\citeANP{#1}, \citeyearNP{#1}}
\def\citetwocomma#1#2{\citeANP{#1}, \citeyearNP{#1}, \citeyearNP{#2}}
\def\citethree#1#2#3{\citeANP{#1} (\citeyearNP{#1}, \citeyearNP{#2}, \citeyearNP{#3})}
\def\citetwo#1#2#3{\citeANP{#1} (\citeyearNP{#1}, \citeyearNP{#2})}

\xdefinecolor{mediumred}{rgb}{0.8 0 0}                   % this is a darker version of red
\xdefinecolor{mediumgreen}{rgb}{0.1 0.5 0.1}                   % this is a darker version of red
\xdefinecolor{lightblue}{rgb}{0 1 1}
\xdefinecolor{lightred}{rgb}{1 0.5 0.5}
\xdefinecolor{lightgreen}{rgb}{0.5 1 0.5}

\def\baselinestretch{1.2}
\def\arraystretch{1.2}

\graphicspath{{./figure/}{../figure/}{./}}

\def\doRed#1{{\it #1}}
\def\doDarkgreen#1{\textbf{#1}}
\def\doBlue#1{#1}

\def\Figure#1#2{
\begin{figure}[!tb]
\begin{center}
\includegraphics[scale=1.1]{#1.pdf}
\vspace{-5ex}\caption{\small #2}
\label{#1}
\end{center}
\end{figure}
}
\def\Figref#1{Figure~\ref{#1}}

\begin{document}

\title{M4 Card forecasts}

\author{Jurgen A.\ Doornik \\
Economics Department, Nuffield college, and Institute for New Economic Thinking\\
at the Oxford Martin School, University of Oxford, UK.}
\maketitle

\vspace{2ex}


\section{Introduction}

This codes replicates the submission of the \textit{Card} forecasts for the M4 competition,
see Doornik, Castle, Hendry \textit{(reference coming later)}.

The code requires Ox 7 or newer, see \cite{Ox}.

Running \texttt{code/forecast\_card.ox} will create the submission files
\texttt{M4\_Doornik\_Card.csv},\\
\texttt{M4\_Doornik\_CardHI.csv}, \texttt{M4\_Doornik\_CardLO.csv}, in
the \texttt{code} folder. These should be identical to the files with \texttt{submitted} prefix.

\textbf{NB1. The M4 data is the competition version: data have been withheld by the organizers
and are not available at this stage.} However, the M3 data is the full data set.

If the M4 data is extended, the following code would allow replication of the submitted results
\begin{verbatim}
    decl ch = m4.GetH();
    db.SetCutBack(ch);
    db.SetHoldBack(ch);
\end{verbatim}
assuming \texttt{m4} is a valid M4 object and \texttt{db} a ForecastAB object.

\textbf{NB2. The focus is on replication at this stage.} If there is more interest when the results
of the competition have become available, we will provide formal documentation that can be cited, as well as Ox code.


\section{\texttt{data} folder}

\subsection{Creating M4 data sets for analysis with OxMetrics}
\label{se:data}

\begin{description}
\item[data\_original] The M4 data is supplied in csv files. However, these have the observations as a text string, which
would need conversion first for handling in Ox. To fix this, load the files are in Excel, and save as xlsx files:
\texttt{Daily-train.xlsx} ... \texttt{Yearly-train.xlsx}.

This folder and the original data is not included here.

\item[data] \texttt{convert\_m4\_data.ox} reads the xlsx files, and creates pairs of \texttt{in7/bn7} files.
This format can be read very quickly by OxMetrics and Ox.

Because the variables are unbalanced, data files organized just by frequency would have many missing values, and so be slow to load.
For this reason the files are somewhat arbitrarily split in chunks, after sorting the variables by sample size by name:

\begin{center}
\begin{tabular}{lrrrrrrrr}
\toprule
    & \# files & \# series & \%  &$H$ & $T_{\text{min}}$& $T_{\text{max}}$ \\
\midrule
Hourly       & 1  &  414&  0.4&     48& 700&   960\\
Daily        & 7  & 4227&  4.2&     14&  93&  9919\\
Weekly       & 1  &  359&  0.4&     13&  80&  2597\\
Monthly      & 11 &48000& 48.0&     18&  42&  2794\\
Quarterly    & 5  &24000& 24.0&      8&  16&   866\\
Yearly       & 6  &23000& 23.0&      6&  13&   835\\
\bottomrule
\end{tabular}
\end{center}

In contrast to the original files, the observations are aligned at the end, in such a way that the first forecast
is always for 2000(1). The databases have $H$ missing values at the end, reflecting the forecast horizon.

Because the variables are sorted by size, at the end, when the submission is generated, the code has to
sort them back into the original numerical order.

\end{description}

\subsection{Data handling}
\label{se:handling}

The data for a frequency are read into an array of databases. An analysis involves a loop over databases, and
within databases over all variables. This processes the variables in increasing sample size (and, for the same sample size,
by variable name).

\subsection{Data sample}

Data is given on an isolated time series $y_t$, and the objective is to forecast $y_t$.
In the remainder, $y_t$ is always the original series, while $x_t$ is a (possibly) transformed version.

The sample may contain missing observations (but not in M4), so is decomposed in:
\begin{center}
\begin{tabular}{ll}
$T_0$& first valid observation,                  \\
$T_2$& last valid observation,                   \\
$T_1$& start of the trailing contiguous block,   \\
$H$& number of forecasts.
\end{tabular}
\end{center}

The data available as the basis for forecasting has $T=T_2-T_1+1$ observations. Forecasting is from $T_2+1$ onwards.
$T_2$ may be reduced if data is held back for forecasting, while $T_1$ is increased if the sample is deemed `too long.'
The $T_0,...,T_1$ part of the sample has missing values, unless $T_1=T_0$.

The primary frequency $S$ is set in the database. If there is no seasonality at $S$, but constant
seasonality is found at a lower frequency, then this can be adopted. A second factor $S_2$ may be used
to introduce seasonality at $SS_2$. E.g. hourly data has $S=24, S_2 = 7$ for a weekly frequency of 168.

\section{\texttt{data\_M3} folder}

For comparison, the M3 data is supplied in the same format as M4. So it is easy to switch from one to the other.

M3 needs only one file each for annual, querterly and monthly data. The `other' data is not included,
because it was not clear to me what the appropriate frequency should be.

\section{\texttt{code} folder}

\begin{itemize}
\item \texttt{evaluate\_insample.ox} Evaluates forecast methods by withholding $H$ observations from the M4 data.

Switch to M3 by setting \texttt{ism3} in \texttt{main} to 1. Use \texttt{astypes} in the \texttt{foreach}
loop to run over all frequencies. The current version prints (truncated):
{\small\begin{verbatim}
loaded ../data/Yearly_01.in7 with 4556 series
loaded ../data/Yearly_02.in7 with 4997 series
loaded ../data/Yearly_03.in7 with 4387 series
loaded ../data/Yearly_04.in7 with 5084 series
loaded ../data/Yearly_05.in7 with 3621 series
loaded ../data/Yearly_06.in7 with 355 series
Run done in  3.92

        MAPE(naive2) MAPE(Delta) MAPE(Rho)MAPE(Delta+C) MAPE(Rho+C) MAPE(Card)
Yearly-median 13.923       8.212     8.882        8.162       8.650      8.183
Yearly-mean   18.462      15.152    15.485       15.112      15.245     14.886
Yearly-mean    1.000       0.821     0.839        0.819       0.826      0.806
\end{verbatim}}


\item \texttt{experiment.ox} Creates forecasts for some specific series.

This program is intended for experimenting or considering a specific series. 

\item \texttt{forecast\_card.ox} Creates our M4 submission, producing out-of-sample \textit{Card} forecasts.

\item \texttt{ForecastAB.oxo} Forecast framework and forecasting methods.
\item \texttt{M4.ox} Database management for M4 data.
\end{itemize}

\section{\texttt{code} folder}

Contains this document, as well as oxdoc documentation for the \texttt{ForecastAB} and \texttt{M4} classes.

\small
\renewcommand{\baselinestretch}{0.9}
\bibliographystyle{chicago}
\bibliography{journals,jurgen,dynects,david}


\end{document}
